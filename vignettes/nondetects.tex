% automatic manuscript creation for nondetects
% -*- mode: noweb; noweb-default-code-mode: R-mode; -*-
%\VignetteIndexEntry{nondetects - vignette}
%\VignetteDepends{nondetects, HTqPCR}
%\VignettePackage{nondetects}
\documentclass[12pt]{article}
\usepackage{hyperref, Sweave}
%\usepackage{natbib}

\textwidth=6.2in
\textheight=8.5in
\parskip=.3cm
\oddsidemargin=.1in
\evensidemargin=.1in
\headheight=-.3in

\newcommand\Rpackage[1]{{\textsf{#1}\index{#1 (package)}}}
\newcommand\dataset[1]{{\textit{#1}\index{#1 (data set)}}}
\newcommand\Rclass[1]{{\textit{#1}\index{#1 (class)}}}
\newcommand\Rfunction[1]{{{\small\texttt{#1}}\index{#1 (function)}}}
\newcommand\Rfunarg[1]{{\small\texttt{#1}}}
\newcommand\Robject[1]{{\small\texttt{#1}}}

\author{Matthew N. McCall}

\begin{document}
\title{Non-detects in qPCR data: methods to model and impute
  non-detects in the results of qPCR experiments (nondetects)}
\maketitle \tableofcontents

\section{Background on non-detects in qPCR data}
Quantitative real-time PCR (qPCR) measures gene expression for a
subset of genes through repeated cycles of sequence-specific DNA
amplification and expression measurements. During the exponential
amplification phase, each cycle results in an approximate doubling of
the quanitity of each target transcript. The threshold cycle (Ct) --
the cycle at which the target gene's expression first exceeds a
predetermined threshold -- is used to quantify the expression of each
target gene. These Ct values typically represent the raw data from a
qPCR experiment.

One challenge of qPCR data is the presence of \emph{non-detects} --
those reactions failing to attain the expression threshold. While most
current software replaces these non-detects with the maximum possible
Ct value (typically 40), recent work has shown that this introduces
large biases in estimation of both absolute and differential
expression. Here, we treat the non-detects as missing data, model the
missing data mechanism, and use this model to impute Ct values for the
non-detects.

\section{EM algorithm}
We propose the following model of observed expression for gene $i$,
sample-type $j$, and replicate $k$, $Y_{ijk}$:
\begin{displaymath}
Y_{ijk} = \left\{ \begin{array}{ll}
\theta_{ij} + \delta_{k} + \varepsilon_{ijk} & \textrm{if $Z_{ijk}=1$}\\
\textrm{non-detect} & \textrm{if $Z_{ijk}=0$}
\end{array} \right.
\end{displaymath}
where $\delta_{k}$ represents a global shift in expression across samples and,
\begin{displaymath}
Pr(Z_{ijk}=1) = \left\{ \begin{array}{ll}
g(Y_{ijk}) & \textrm{if $Y_{ijk} < 40$} \\
0 & \textrm{otherwise}
\end{array} \right.
\end{displaymath}
Here, $g(Y_{ijk})$ can be estimated via the following logistic regression:
\[
logit(Pr(Z_{ijk}=1)) =  \beta_0 + \beta_1 \hat{\theta}_{ij}
\]
where $\hat{\theta}_{ij}$ is an estimate of the average expression for
gene $i$ and sample-type $j$. 

\section{Example}

\subsection*{Data from Sampson \emph{et al.} Oncogene 2013}
Two cell types -- young adult mouse colon (YAMC) cells and
mutant-p53/activated-Ras transformed YAMC cells -- in combination with
three treatments -- untreated, sodium butyrate, or valproic acid. Four
replicates were performed for each cell-type/treatment combination
\cite{sampson2012gene}.

\subsection*{Load the data}


























